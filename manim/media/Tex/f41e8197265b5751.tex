\documentclass[preview]{standalone}

\usepackage[english]{babel}
\usepackage[utf8]{inputenc}
\usepackage[T1]{fontenc}
\usepackage{lmodern}
\usepackage{amsmath}
\usepackage{amssymb}
\usepackage{dsfont}
\usepackage{setspace}
\usepackage{tipa}
\usepackage{relsize}
\usepackage{textcomp}
\usepackage{mathrsfs}
\usepackage{calligra}
\usepackage{wasysym}
\usepackage{ragged2e}
\usepackage{physics}
\usepackage{xcolor}
\usepackage{microtype}
\DisableLigatures{encoding = *, family = * }
\linespread{1}

\begin{document}

\begin{center}
This chapter presents methods for representing a graph and for searching a graph.Searching a graph means systematically following the edges of the graph so as to visit the vertices of the graph. A graph-searching algorithm can discover much about the structure of a graph. Many algorithms begin by searching their input graph to obtain this structural information. Several other graph algorithms elaborate  on basic graph searching. Techniques for searching a graph lie at the heart of the field of graph algorithms
\end{center}

\end{document}
