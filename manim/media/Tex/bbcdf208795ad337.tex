\documentclass[preview]{standalone}

\usepackage[english]{babel}
\usepackage[utf8]{inputenc}
\usepackage[T1]{fontenc}
\usepackage{lmodern}
\usepackage{amsmath}
\usepackage{amssymb}
\usepackage{dsfont}
\usepackage{setspace}
\usepackage{tipa}
\usepackage{relsize}
\usepackage{textcomp}
\usepackage{mathrsfs}
\usepackage{calligra}
\usepackage{wasysym}
\usepackage{ragged2e}
\usepackage{physics}
\usepackage{xcolor}
\usepackage{microtype}
\DisableLigatures{encoding = *, family = * }
\linespread{1}

\begin{document}

\begin{center}
A potential disadvantage of the adjacency-list representation is that it provides
no quicker way to determine whether a given edge (u, v) is present in the graph
than to search for  in the adjacency list Adj[u]. An adjacency-matrix representation of the graph remedies this disadvantage, but at the cost of using asymptotically
more memory.
\end{center}

\end{document}
