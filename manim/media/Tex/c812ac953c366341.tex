\documentclass[preview]{standalone}

\usepackage[english]{babel}
\usepackage[utf8]{inputenc}
\usepackage[T1]{fontenc}
\usepackage{lmodern}
\usepackage{amsmath}
\usepackage{amssymb}
\usepackage{dsfont}
\usepackage{setspace}
\usepackage{tipa}
\usepackage{relsize}
\usepackage{textcomp}
\usepackage{mathrsfs}
\usepackage{calligra}
\usepackage{wasysym}
\usepackage{ragged2e}
\usepackage{physics}
\usepackage{xcolor}
\usepackage{microtype}
\DisableLigatures{encoding = *, family = * }
\linespread{1}

\begin{document}

\begin{center}
{The procedure BFS builds a breadth-first tree as it searches the graph. The tree corresponds to the $\pi$ attributes. More formally, for a graph G = (V, E) with source s, we define the predecessor subgraph of G asTex('$G_{\\pi}$ = ($V_{\\pi}$, $E_{\\pi}$), where')Tex('$V_{\\pi}$ = {v $\\in$ V : v.$\\pi$ $\\neq$ NIL} $\\cup$ {s}')Tex('and \\\\ $E_{\\pi}$ = {(v.$\\pi$, v} : v $\\in$ $V_{\\pi}$ - {s}}')
\end{center}

\end{document}
