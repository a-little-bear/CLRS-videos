\documentclass[preview]{standalone}

\usepackage[english]{babel}
\usepackage[utf8]{inputenc}
\usepackage[T1]{fontenc}
\usepackage{lmodern}
\usepackage{amsmath}
\usepackage{amssymb}
\usepackage{dsfont}
\usepackage{setspace}
\usepackage{tipa}
\usepackage{relsize}
\usepackage{textcomp}
\usepackage{mathrsfs}
\usepackage{calligra}
\usepackage{wasysym}
\usepackage{ragged2e}
\usepackage{physics}
\usepackage{xcolor}
\usepackage{microtype}
\DisableLigatures{encoding = *, family = * }
\linespread{1}

\begin{document}

\begin{align*}
The adjacency-list representation of a graph G = (V, E) consists of an array Adj of |V| lists, \n
one for each vertex in V. For each u \in V , \n
the adjacency list Adj[u] contains all the vertices such that there is an edge (u, v) \in E. \n
That is, Adj[u] consists of all the vertices adjacent to u in G.
\end{align*}

\end{document}
