\documentclass[preview]{standalone}

\usepackage[english]{babel}
\usepackage[utf8]{inputenc}
\usepackage[T1]{fontenc}
\usepackage{lmodern}
\usepackage{amsmath}
\usepackage{amssymb}
\usepackage{dsfont}
\usepackage{setspace}
\usepackage{tipa}
\usepackage{relsize}
\usepackage{textcomp}
\usepackage{mathrsfs}
\usepackage{calligra}
\usepackage{wasysym}
\usepackage{ragged2e}
\usepackage{physics}
\usepackage{xcolor}
\usepackage{microtype}
\DisableLigatures{encoding = *, family = * }
\linespread{1}

\begin{document}

\begin{center}
The breadth-first seach procedure BFS below assume that the input graph G = (V, E) is represented using adjacency lists. It attaches several additional attributes to each vertex in the graph. We store the color of each vertex u $\in$ V in the attribute u.color and the prodecessor of u in the attribute u.$\pi$. If u has no predecessor (for example, if u = s or u has not been discovered), then u.$\pi$ = NIL. The attribute u.d holds the distance from the source s to vertex u computed by the algorithm. The algorithm also uses a first-in, first-out queue Q to manage the set of vertices.
\end{center}

\end{document}
