\documentclass[preview]{standalone}

\usepackage[english]{babel}
\usepackage[utf8]{inputenc}
\usepackage[T1]{fontenc}
\usepackage{lmodern}
\usepackage{amsmath}
\usepackage{amssymb}
\usepackage{dsfont}
\usepackage{setspace}
\usepackage{tipa}
\usepackage{relsize}
\usepackage{textcomp}
\usepackage{mathrsfs}
\usepackage{calligra}
\usepackage{wasysym}
\usepackage{ragged2e}
\usepackage{physics}
\usepackage{xcolor}
\usepackage{microtype}
\DisableLigatures{encoding = *, family = * }
\linespread{1}

\begin{document}

\begin{center}
Breadth-first search constructs a breadth-first tree, initially containing only its root, which is the source vertex s. Whenever the search discovers a while vertex v in the course of scanning the adjacency list of an already discovered vertex u, the vertex v and the edge (u, v) are added to the tree. We say that u isi the predecessor of parent of v in the breadth-first tree. Since a vertex is discovered at most one, it has at most one parent. Ancestor and descendant relationships in the breadth-first tree are definied relative to the root s as usual: if u is on the simple path in the tree from the root s to vertex v, then u is an ancestor of v and v is a descendant of u.
\end{center}

\end{document}
