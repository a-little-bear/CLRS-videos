\documentclass[preview]{standalone}

\usepackage[english]{babel}
\usepackage[utf8]{inputenc}
\usepackage[T1]{fontenc}
\usepackage{lmodern}
\usepackage{amsmath}
\usepackage{amssymb}
\usepackage{dsfont}
\usepackage{setspace}
\usepackage{tipa}
\usepackage{relsize}
\usepackage{textcomp}
\usepackage{mathrsfs}
\usepackage{calligra}
\usepackage{wasysym}
\usepackage{ragged2e}
\usepackage{physics}
\usepackage{xcolor}
\usepackage{microtype}
\DisableLigatures{encoding = *, family = * }
\linespread{1}

\begin{document}

\begin{center}
Like the adjacency-list representation of a graph, an adjacency matrix can represent a weighted graph. For example, if G = (V, E) is a weighted graph with edge-weight function w, we can simply store the weight w(u, v) of the edge (u, v) $\in$ E as the entry in row u and column v of the adjacency matrix. If an edge does not exist, we can sotre a NIL value as its corresponding matrix entry, though for many problems it is convenient to use a value such as 0 or $\infty$.
\end{center}

\end{document}
