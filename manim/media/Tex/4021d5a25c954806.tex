\documentclass[preview]{standalone}

\usepackage[english]{babel}
\usepackage[utf8]{inputenc}
\usepackage[T1]{fontenc}
\usepackage{lmodern}
\usepackage{amsmath}
\usepackage{amssymb}
\usepackage{dsfont}
\usepackage{setspace}
\usepackage{tipa}
\usepackage{relsize}
\usepackage{textcomp}
\usepackage{mathrsfs}
\usepackage{calligra}
\usepackage{wasysym}
\usepackage{ragged2e}
\usepackage{physics}
\usepackage{xcolor}
\usepackage{microtype}
\DisableLigatures{encoding = *, family = * }
\linespread{1}

\begin{document}

\begin{center}
At the beginning of this section, we claimed that breadth-first search finds the distance to each reachable vertex in a graph G = (V, E) from a given source vertex s $\in$ V . Define the shortest-path distance 8(s, v) from s to v as the minimum number of edges in any path from vertex s to vertex ; if there is no path from s to v,then 8(s, v) = $\infty$. We call a path of length 8(s, v) from s to v a shortest path from s to v. Before showing that breadth-first search correctly computes shortestpath distances, we investigate an important property of shortest-path distances.
\end{center}

\end{document}
